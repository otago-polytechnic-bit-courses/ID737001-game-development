% Author: Grayson Orr
% Course: ID607001: Introductory Application Development Concepts

\documentclass{article}
\author{}
 
\usepackage{fontspec}
\setmainfont{Arial}

\usepackage{graphicx}
\usepackage{wrapfig}
\usepackage{enumerate}
\usepackage{hyperref}
\usepackage[margin = 2.25cm]{geometry}
\usepackage[table]{xcolor}
\usepackage{soul}
\usepackage{fancyhdr}
\hypersetup{
  colorlinks = true,
  urlcolor = blue
} 
\setlength\parindent{0pt}
\pagestyle{fancy}
\fancyhf{}
\rhead{College of Engineering, Construction and Living Sciences\\Bachelor of Information Technology}
\lfoot{Project\\Version 1, Semester One, 2024}
\rfoot{\thepage}  
 
\begin{document}

\begin{figure}
	\centering
	\includegraphics[width=50mm]{../../img/logo.png}
\end{figure}

\title{College of Engineering, Construction and Living Sciences\\Bachelor of Information Technology\\ID607001: Introductory Application Development Concepts\\Level 6, Credits 15\\\textbf{Project}}
\date{}
\maketitle

\section*{Assessment Overview}
In this \textbf{individual} assessment, you will design and develop a game using \textbf{Unreal Engine}. In addition, marks will be allocated for code quality and best practices, documentation and Git usage.

\section*{Learning Outcome}
At the successful completion of this course, learners will be able to:
\begin{enumerate}
	\item Design and build secure applications with dynamic database functionality following an appropriate software development methodology.
\end{enumerate}

\section*{Assessments}
\renewcommand{\arraystretch}{1.5}
\begin{tabular}{|c|c|c|c|}
	\hline
	\textbf{Assessment}                                 & \textbf{Weighting} & \textbf{Due Date}            & \textbf{Learning Outcome} \\ \hline
	\small Practical & \small 20\%        & \small 21-06-2024 (Friday at 4.59 PM)   & \small 1                   \\ \hline
	\small Project                 & \small 80\%        & \small 21-06-2024 (Friday at 4.59 PM) \small  & \small 1                   \\ \hline
\end{tabular}

\section*{Conditions of Assessment}
You will complete this assessment during your learner-managed time. However, there will be time during class to discuss the requirements and your progress on this assessment. This assessment will need to be completed by \textbf{Friday, 21 June 2024} at \textbf{4.59 PM}. 

\section*{Pass Criteria}
This assessment is criterion-referenced (CRA) with a cumulative pass mark of \textbf{50\%} across all assessments in \textbf{ID607001: Introductory Application Development Concepts}.

\section*{Submission}
You \textbf{must} submit all application files via \textbf{GitHub Classroom}. Here is the URL to the repository you will use for your submission – \href{https://classroom.github.com/a/wlzE5yYo}{https://classroom.github.com/a/wlzE5yYo}. If you do not have not one, create a \textbf{.gitignore}. The latest application files in the \textbf{main} branch will be used to mark against the \textbf{Technical and Professional Proficiency} criterion. Please test before you submit. Partial marks \textbf{will not} be given for incomplete functionality. Late submissions will incur a \textbf{10\% penalty per day}, rolling over at \textbf{5:00 PM}.

\section*{Authenticity}
All parts of your submitted assessment \textbf{must} be completely your work. Do your best to complete this assessment without using an \textbf{AI generative tool}. You need to demonstrate to the course lecturer that you can meet the learning outcome for this assessment. \\
 
 However, if you get stuck, you can use an \textbf{AI generative tool} to help you get unstuck, permitting you to acknowledge that you have used it. In the assessment's repository \textbf{README.md} file, please include what prompt(s) you provided to the \textbf{AI generative tool} and how you used the response(s) to help you with your work. It also applies to code snippets retrieved from \textbf{StackOverflow} and \textbf{GitHub}. \\ 
 
 Failure to do this may result in a mark of \textbf{zero} for this assessment.

\section*{Policy on Submissions, Extensions, Resubmissions and Resits}
The school's process concerning submissions, extensions, resubmissions and resits complies with \textbf{Otago Polytechnic | Te Pūkenga} policies. Learners can view policies on the \textbf{Otago Polytechnic | Te Pūkenga} website located at \href{https://www.op.ac.nz/about-us/governance-and-management/policies}{https://www.op.ac.nz/about-us/governance-and-management/policies}. 

\section*{Extensions}
Familiarise yourself with the assessment due date. Extensions will \textbf{only} be granted if you are unable to complete the assessment by the due date because of \textbf{unforeseen circumstances outside your control}. The length of the extension granted will depend on the circumstances and \textbf{must} be negotiated with the course lecturer before the assessment due date. A medical certificate or support letter may be needed. Extensions will not be granted for poor time management or pressure of other assessments.

\section*{Resits}
Resits and reassessments are not applicable in \textbf{ID607001: Introductory Application Development Concepts}. 

\newpage

\section*{Instructions}

\subsection*{Technical and Professional Proficiency - Learning Outcome 1 (50\%)}
\begin{itemize} 
	\item The topic for the game is \textbf{your choice}.  
	\item The game needs to open without code or file structure modification in \textbf{Unreal Engine}.
	\item Gather requirements from the client and deconstruct them into user stories.
	\item Design and develop a game using \textbf{Unreal Engine}.
	\item Integrate a \textbf{database} into the game. The type of \textbf{database} is \textbf{your choice}.
	\item Demo the game on a web platform.
\end{itemize}

\subsection*{Code Quality and Best Practices - Learning Outcome 1 (30\%)}
\begin{itemize}
    \item An appropriate \textbf{.gitignore} file is used. 
    \item Appropriate naming of files, variables, methods and classes.
    \item Idiomatic use of values, control flow, data structures and in-built functions.
    \item Efficient algorithmic approach.
    \item Sufficient modularity.
    \item Each file has an \textbf{comment} located at the top of the file. 
    \item Formatted code.
    \item No dead or unused code.
\end{itemize} 

\subsection*{Documentation and Git Usage - Learning Outcome 1 (20\%)}
\begin{itemize}
	\item \textbf{GitHub} issues and a project board to help you organise and prioritise your development work. The course lecturer needs to see consistent use of \textbf{GitHub} issues and the project board for the duration of the assessment.
	\item In a \textbf{Microsoft Word} document called \textbf{game-document}, explain the following:
	\begin{itemize}
		\item Core concept
		\item Design pillars
		\item Main features and mechanics
		\item Target platform and audience
		\item Interface and controls
		\item Basic story
		\item Visual style
		\item Audio style
		\item Security considerations
		\item Known issues and bugs
		\item Future improvements
	\end{itemize}
    \item Correct spelling and grammar.
    \item Your \textbf{Git commit messages} should:
    \begin{itemize}
      \item Reflect the context of each functional requirement change.
      \item Be formatted using an appropriate naming convention style.
    \end{itemize}
\end{itemize} 

\subsection*{Additional Information}
\begin{itemize}
    \item \textbf{Do not} rewrite your \textbf{Git} history. It is important that the course lecturer can see how you worked on your assessment over time. 
    \item You need to show the course lecturer the initial \textbf{GitHub} project board or issues before you start your development work. Following this, you need to show the course lecturer your \textbf{GitHub} project board or issues at the end of each week.
\end{itemize} 
\end{document}