% Author: Grayson Orr
% Course: ID737001: Game Development

\documentclass{article}
\author{}

\usepackage{fontspec}
\setmainfont{Arial}

\usepackage{graphicx}
\usepackage{wrapfig}
\usepackage{enumerate}
\usepackage{hyperref}
\usepackage[margin = 2.25cm]{geometry}
\usepackage[table]{xcolor}
\usepackage{fancyhdr}
\hypersetup{
  colorlinks = true,
  urlcolor = blue
}
\setlength\parindent{0pt}
\pagestyle{fancy}
\fancyhf{}
\rhead{College of Engineering, Construction and Living Sciences\\Bachelor of Information Technology}
\lfoot{Project: Game Development + Demo\\Version 2, Semester One, 2025}
\rfoot{\thepage}
 
\begin{document}

\begin{figure}
	\centering
	\includegraphics[width=50mm]{../../resources/img/logo.jpg}
\end{figure}

\title{College of Engineering, Construction and Living Sciences\\Bachelor of Information Technology\\ID737001: Game Development\\Level 7, Credits 15\\\textbf{Project: Game Development + Demo}}
\date{}
\maketitle

\section*{Assessment Overview}
In this assessment, you will:
\begin{itemize}
	\item Form a \textbf{group of three} to design and develop a game using a game engine of your choice. In addition, marks will be allocated for code quality and best practices, documentation and Git usage.
	\item Research, analyse and evaluate a game, producing a game analysis report covering a range of key ideas. The game can be of your choice. 
\end{itemize}

\section*{Learning Outcome}
At the successful completion of this course, learners will be able to:
\begin{enumerate}
	\item Design and develop a game using industry standard tools, technologies and practices.
\end{enumerate}

\section*{Assessments}
\renewcommand{\arraystretch}{1.5}
\begin{tabular}{|c|c|c|c|}
	\hline
	\textbf{Assessment}                                 & \textbf{Weighting} & \textbf{Due Date}            & \textbf{Learning Outcome} \\ \hline
	\small Assignment                 & \small 30\%        & \small Monday 9th June at 7.59 AM \small  & \small 1                   \\ \hline
	\small Project: Game Development + Demo                 & \small 70\%        & \small Monday 9th June at 7.59 AM \small  & \small 1                   \\ \hline
\end{tabular}

\section*{Conditions of Assessment}
You will complete this assessment during your learner-managed time. However, there will be time during class to discuss the requirements and your progress on this assessment. This assessment will need to be completed by \textbf{Monday, 9 June 2025} at \textbf{7.59 AM}. 

\section*{Pass Criteria}
This assessment is criterion-referenced (CRA) with a cumulative pass mark of \textbf{50\%} over all assessments in \textbf{ID737001: Game Development}.

\section*{Authenticity}
All parts of your submitted assessment \textbf{must} be completely your work. Do your best to complete this assessment without using an \textbf{AI generative tool}. You need to demonstrate to the course lecturer that you can meet the learning outcome for this assessment. \\
 
Learning to use AI tool is an important skill. While AI tools are powerful, you \textbf{must} be aware of the following:

\begin{itemize}
    \item If you provide an AI tool with a prompt that is not refined enough, it may generate a not-so-useful response
    \item Do not trust the AI tool's responses blindly. You \textbf{must} still use your judgement and may need to do additional research to determine if the response is correct
    \item Acknowledge what AI tool you have used. In the assessment's repository \textbf{README.md} file, please include what prompt(s) you provided to the AI tool and how you used the response(s) to help you with your work
\end{itemize}

It also applies to code snippets retrieved from \textbf{StackOverflow} and \textbf{GitHub}. \\
 
Failure to do this may result in a mark of \textbf{zero} for this assessment.

\section*{Policy on Submissions, Extensions, Resubmissions and Resits}
The school's process concerning submissions, extensions, resubmissions and resits complies with \textbf{Otago Polytechnic} policies. Learners can view policies on the \textbf{Otago Polytechnic} website located at \href{https://www.op.ac.nz/about-us/governance-and-management/policies}{https://www.op.ac.nz/about-us/governance-and-management/policies}.

\section*{Submission}
You \textbf{must} submit all application files via \textbf{GitHub}. Create a repository and add the course lecturer as a collaborator. Late submissions will incur a \textbf{10\% penalty per day}, rolling over at \textbf{5:00 PM}.

\section*{Extensions}
Familiarise yourself with the assessment due date. Extensions will \textbf{only} be granted if you are unable to complete the assessment by the due date because of \textbf{unforeseen circumstances outside your control}. The length of the extension granted will depend on the circumstances and \textbf{must} be negotiated with the course lecturer before the assessment due date. A medical certificate or support letter may be needed. Extensions will not be granted for poor time management or pressure of other assessments.

\section*{Resits}
Resits and reassessments \textbf{are not} applicable in \textbf{ID737001: Game Development}.

\section*{Instructions}

\subsection*{Project - Technical and Professional Proficiency (Individual and Group) - Learning Outcome 1 (45\%)}
\begin{itemize}
	 \item \textbf{Group:}
	\begin{itemize}
		\item The game needs to open without code or file structure modification in the chosen game engine.
		\item Gather requirements and deconstruct them into user stories.
		\item Design and develop a game using the chosen game engine that meets the requirements. 
		\item Demo the game on \textbf{itch.io}.
	\end{itemize}
	\item \textbf{Individual:}
	\begin{itemize}
		\item Contribute a meaningful amount of code to the game. This will be judged by the number of \textbf{Git commits} and the number of lines of code contributed.
		\item Perform the following for each feature that is merged into the \textbf{main} branch of the \textbf{GitHub} repository:
		\begin{itemize}
			\item Code review another team member's code.
			\item Play test the feature and provide feedback to the team member.
		\end{itemize}
		This needs to be documented in the \textbf{GitHub} issue that the feature is associated with.
		\item Communicate with team members. This should be through \textbf{Microsoft Teams}. If you wish to use another communication tool, you need to get approval from the course lecturer. Provide screenshots of your communication in the \textbf{GitHub} repository.
	\end{itemize}
\end{itemize}

\subsection*{Project - Code Quality and Best Practices (Individual) - Learning Outcome 1 (15\%)}
\begin{itemize}
    \item Appropriate naming of files, variables, methods and classes.
    \item Idiomatic use of the programming language and game engine.
    \item Efficient algorithmic approach.
    \item Sufficient modularity considering the \textbf{SOLID principles} and \textbf{design patterns}.
    \item Each file has a \textbf{comment} located at the top of the file. The comment should explain the purpose of the file and the author.
    \item Formatted code.
    \item No dead or unused code.
\end{itemize} 

\subsection*{Project - Git Usage (Individual and Group) - Learning Outcome 1 (10\%)}
\begin{itemize}
	\item \textbf{Group} requirement - \textbf{GitHub} project board or issues to help you organise and prioritise your development work. The course lecturer needs to see consistent use of \textbf{GitHub} issues and the project board for the duration of the assessment. 
	\item \textbf{Individual} requirement - Your \textbf{Git commit messages} should:
    \begin{itemize}
      \item Reflect the context of each functional requirement change.
      \item Be formatted using an appropriate naming convention style. 
    \end{itemize}
\end{itemize}

\subsection*{Project - Documentation (Individual and Group) - Learning Outcome 1 (15\%)}
\begin{itemize}
	\item \textbf{Group} requirement - In a \textbf{Microsoft Word} document, explain the following:
	\begin{itemize}
		\item Basic story
		\item Design pillars
		\item Gameplay
		\item Main features and mechanics
		\item Target platform and audience
		\item Interface and controls
		\item Inspiration
		\item Visual style
		\item Audio style
		\item Known issues and bugs
		\item Future improvements
		\item A URL to the game on \textbf{itch.io}.
	\end{itemize}
	\item \textbf{Group} requirement - Engage with \textbf{five external} play testers and in a \textbf{Microsoft Word} document, record the following:
	\begin{itemize}
		\item Overall experience:
		\begin{itemize}
			\item Rate your overall experience playing the game from 0-5 (0 being the worst and 5 being the best).
			\item A brief explanation of your experience. Highlight one thing you enjoyed and one thing you did not enjoy.
		\end{itemize}
		\item Game mechanics:
		\begin{itemize}
			\item Identify any game mechanics that felt intuitive or unintuitive.
			\item Improvements to enhance the game mechanics.
		\end{itemize}
		\item Controls:
		\begin{itemize}
			\item Were the controls easy to learn and use?
			\item Did you encounter any issues with the controls?
		\end{itemize}
		\item User interface:
		\begin{itemize}
			\item Was important information presented clearly?
			\item Did the user interface enhance or detract from the game?
		\end{itemize}
		\item Difficulty:
		\begin{itemize}
			\item Was the game too easy, too hard or just right?
			\item A brief explanation of challenges that felt challenging or unfair. 
		\end{itemize}  
		\item Bugs:
		\begin{itemize}
			\item Document any bugs you encountered during play testing.
		\end{itemize}
	\end{itemize}
	\item \textbf{Individual} requirement - Select three interesting game mechanics with that you implemented and in a \textbf{Microsoft Word} document the following:
		\begin{itemize}
			\item What did you implement?
			\item What did you research during the implementation? Provide a link to the resources you used.
			\item What did you try? What worked? What did not work?
			\item What did you learn?
			\item How can you apply what you learned to future games?	
			\item In addition, what did you find most challenging professionally? How did you overcome it?	
		\end{itemize}	
    \item \textbf{Individual} and \textbf{group} requirement - Correct spelling and grammar.
	\item \textbf{Individual} and \textbf{group} requirement - References and in-line citations are formatted using APA 7th edition.
\end{itemize}

\subsection*{Game Analysis Document (Individual) - Learning Outcome 1 (15\%)}
\begin{itemize}
	\item Overview of the game. \textbf{Note:} Explain the game to the reader so they have an understanding of what the gameplay is like.
	\begin{itemize}
		\item Number of players, i.e., single-player or multi-player.
		\begin{itemize}
			\item If multi-player, how many players can you have?
			\item Is the game real-time or turned-based?
			\item Are there different gameplay modes based on the number of players?
		\end{itemize}
		\item Rules and goals.
		\begin{itemize}
			\item What are the basic rules of the game?
			\item Is there a win, end or endless state to the game?
			\item What are the goals of the game?
			\item Are there multiple goals in the game?
			\item Can a player set their own goals?
		\end{itemize}
		\item Gameplay.
		\begin{itemize}
			\item What is the genre of the game?
			\item What are the core game mechanics?
			\item What is the player experience like, i.e., satisfaction, learning, efficiency, immersion, motivation,
			emotion and socialisation?
			\item Are there difficulty levels? Does the difficulty level increase as a player progresses through the
			game?
			\item Is the game easy to pick up and play?
		\end{itemize}
		\item Ethics and morals.
		\begin{itemize}
			\item Does the game offer a player ethical and moral decisions of their own, i.e., a player can become a
			hero or villain?
			\item What are ethical and moral decisions in the game that a player can apply to the real world?
		\end{itemize}
	\end{itemize}
	\item Correct spelling and grammar.
	\item References and in-line citations are formatted using APA 7th edition.
\end{itemize}

\subsection*{Additional Information}
\begin{itemize}
    \item \textbf{Do not} rewrite your \textbf{Git} history. It is important that the course lecturer can see how you worked on your assessment over time. 
    \item You need to show the course lecturer the initial \textbf{GitHub} project board or issues before you start your development work. Following this, you need to show the course lecturer your \textbf{GitHub} project board or issues at the end of each week.
\end{itemize} 
\end{document}