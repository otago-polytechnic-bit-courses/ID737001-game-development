% Author: Grayson Orr
% Course: ID737001: Game Development

\documentclass{article}
\author{}

\usepackage{fontspec}
\setmainfont{Arial}

\usepackage{graphicx}
\usepackage{wrapfig}
\usepackage{enumerate}
\usepackage{hyperref}
\usepackage[margin = 2.25cm]{geometry}
\usepackage[table]{xcolor}
\usepackage{fancyhdr}
\hypersetup{
  colorlinks = true,
  urlcolor = blue
}
\setlength\parindent{0pt}
\pagestyle{fancy}
\fancyhf{}
\rhead{College of Engineering, Construction and Living Sciences\\Bachelor of Information Technology}
\lfoot{Project: Game Development + Demo\\Version 1, Semester One, 2024}
\rfoot{\thepage}
 
\begin{document}

\begin{figure}
	\centering
	\includegraphics[width=50mm]{../../resources/img/logo.png}
\end{figure}

\title{College of Engineering, Construction and Living Sciences\\Bachelor of Information Technology\\ID737001: Game Development\\Level 7, Credits 15\\\textbf{Project: Game Development + Demo}}
\date{}
\maketitle

\section*{Assessment Overview}
In this assessment, you will form a \textbf{group of two or three} to design and develop \textbf{two games} using a game engine of your choice. In addition, marks will be allocated for code quality and best practices, documentation and Git usage.

\section*{Learning Outcome}
At the successful completion of this course, learners will be able to:
\begin{enumerate}
	\item Design and develop a game using industry standard tools, technologies and practices.
\end{enumerate}

\section*{Assessments}
\renewcommand{\arraystretch}{1.5}
\begin{tabular}{|c|c|c|c|}
	\hline
	\textbf{Assessment}                                 & \textbf{Weighting} & \textbf{Due Date}            & \textbf{Learning Outcome} \\ \hline
	\small Assignment  & \small 30\%        & \small 07-06-2024 (Friday at 4.59 PM)   & \small 1                  \\ \hline
	\small Project: Game Development + Demo & \small 70\%        & \small 21-06-2024 (Friday at 4.59 PM)   & \small 1                   \\ \hline
\end{tabular} 

\section*{Conditions of Assessment}
You will complete this assessment during your learner-managed time. However, there will be time during class to discuss the requirements and your progress on this assessment. This assessment will need to be completed by \textbf{Friday, 21 June 2024} at \textbf{4.59 PM}. 

\section*{Pass Criteria}
This assessment is criterion-referenced (CRA) with a cumulative pass mark of \textbf{50\%} over all assessments in \textbf{ID737001: Game Development}.

\section*{Authenticity}
All parts of your submitted assessment \textbf{must} be completely your work. Do your best to complete this assessment without using an \textbf{AI generative tool}. You need to demonstrate to the course lecturer that you can meet the learning outcome(s) for this assessment. \\
 
 However, if you get stuck, you can use an \textbf{AI generative tool} to help you get unstuck, permitting you to acknowledge that you have used it. In the assessment's repository \textbf{README.md} file, please include what prompt(s) you provided to the \textbf{AI generative tool} and how you used the response(s) to help you with your work. It also applies to code snippets retrieved from \textbf{StackOverflow} and \textbf{GitHub}. \\
 
 Failure to do this may result in a mark of \textbf{zero} for this assessment.

\section*{Policy on Submissions, Extensions, Resubmissions and Resits}
The school's process concerning submissions, extensions, resubmissions and resits complies with \textbf{Otago Polytechnic} policies. Learners can view policies on the \textbf{Otago Polytechnic} website located at \href{https://www.op.ac.nz/about-us/governance-and-management/policies}{https://www.op.ac.nz/about-us/governance-and-management/policies}.

\section*{Submission}
You \textbf{must} submit all application files via \textbf{GitHub Classroom}. Here is the URL to the repository you will use for your submission – \href{https://classroom.github.com/a/dbrTesMQ}{https://classroom.github.com/a/dbrTesMQ}. If you do not have not one, create a \textbf{.gitignore} and add the ignored files in this resource - \href{https://raw.githubusercontent.com/github/gitignore/main/Unity.gitignore}{https://raw.githubusercontent.com/github/gitignore/main/Unity.gitignore}. The latest application files in the \textbf{main} branch will be used to mark against the \textbf{Functionality} criterion. Please test before you submit. Partial marks \textbf{will not} be given for incomplete functionality. Late submissions will incur a \textbf{10\% penalty per day}, rolling over at \textbf{5:00 PM}.

\section*{Extensions}
Familiarise yourself with the assessment due date. Extensions will \textbf{only} be granted if you are unable to complete the assessment by the due date because of \textbf{unforeseen circumstances outside your control}. The length of the extension granted will depend on the circumstances and \textbf{must} be negotiated with the course lecturer before the assessment due date. A medical certificate or support letter may be needed. Extensions will not be granted for poor time management or pressure of other assessments.

\section*{Resits}
Resits and reassessments \textbf{are not} applicable in \textbf{ID737001: Game Development}.

\section*{Instructions}

\subsection*{Technical and Professional Proficiency (Group) - Learning Outcome 1 (50\%)}

\begin{itemize}
	 \item The topic for the first game jam is \textbf{user choice} and the second game jam is \textbf{escape}. 
	 \item Group:
	\begin{itemize}
		\item The applications needs to open without code or file structure modification in the chosen game engine.
		\item Gather requirements and deconstruct them into user stories.
		\item Design and develop games using the chosen game engine that meets the requirements. 
		\item Demo the games on \textbf{itch.io}.
	\end{itemize}
	\item Individual:
	\begin{itemize}
		\item Contribute a meaningful amount of code to the games. This will be judged by the number of \textbf{Git commits} and the number of lines of code contributed.
		\item Perform the following for each feature that is merged into the \textbf{main} branch of the \textbf{GitHub} repository:
		\begin{itemize}
			\item Code review another team member's code.
			\item Play test the feature and provide feedback to the team member.
		\end{itemize}
		This needs to be documented in the \textbf{GitHub} issue that the feature is associated with.
		\item Communicate with team members. This should be through \textbf{Microsoft Teams}. If you wish to use another communication tool, you need to get approval from the course lecturer. Provide screenshots of your communication in the \textbf{GitHub} repository.
	\end{itemize}
\end{itemize}

\subsection*{Code Quality and Best Practices (Individual) - Learning Outcome 1 (30\%)}
\begin{itemize}
    \item An appropriate \textbf{.gitignore} file is used. 
    \item Appropriate naming of files, variables, methods and classes.
    \item Idiomatic use of values, control flow, data structures and in-built functions.
    \item Efficient algorithmic approach.
    \item Sufficient modularity.
    \item Each file has an \textbf{comment} located at the top of the file. 
    \item Formatted code.
    \item No dead or unused code.
\end{itemize} 

\subsection*{Documentation and Git Usage (Individual and Group) - Learning Outcome 1 (20\%)}
\begin{itemize}
	\item \textbf{Group} requirement - \textbf{GitHub} issues and a project board to help you organise and prioritise your development work. The course lecturer needs to see consistent use of \textbf{GitHub} issues and the project board for the duration of the assessment.
	\item \textbf{Group} requirement - For each game, in a \textbf{Microsoft Word} document, explain the following:
	\begin{itemize}
		\item Core concept
		\item Design pillars
		\item Main features and mechanics
		\item Target platform and audience
		\item Interface and controls
		\item Basic story
		\item Visual style
		\item Audio style
		\item Known issues and bugs
		\item Future improvements
		\item A URL to the game on \textbf{itch.io}.
	\end{itemize}
	\item \textbf{Group} requirement - For each game, engage with \textbf{two external} play testers and in a \textbf{Microsoft Word} document, record the following:
	\begin{itemize}
		\item Overall experience:
		\begin{itemize}
			\item Rate your overall experience playing the game from 0-5 (0 being the worst and 5 being the best).
			\item A brief explanation of your experience. Highlight one thing you enjoyed and one thing you did not enjoy.
		\end{itemize}
		\item Game mechanics:
		\begin{itemize}
			\item Identify any game mechanics that felt intuitive or unintuitive.
			\item Improvements to enhance the game mechanics.
		\end{itemize}
		\item Controls:
		\begin{itemize}
			\item Were the controls easy to learn and use?
			\item Did you encounter any issues with the controls?
		\end{itemize}
		\item User interface:
		\begin{itemize}
			\item Was important information presented clearly?
			\item Did the user interface enhance or detract from the game?
		\end{itemize}
		\item Difficulty:
		\begin{itemize}
			\item Was the game too easy, too hard or just right?
			\item A brief explanation of challenges that felt challenging or unfair. 
		\end{itemize}
		\item Bugs:
		\begin{itemize}
			\item Document any bugs you encountered during play testing.
		\end{itemize}
	\end{itemize}
	\item \textbf{Individual} requirement - For each game, select two interesting game mechanics that you implemented and explain in a \textbf{Microsoft Word} document the following:
		\begin{itemize}
			\item What did you implement?
			\item What did you research during the implementation? Provide a link to the resources you used.
			\item What did you try? What worked? What did not work?
			\item What did you learn?
			\item How can you apply what you learned to future games?	
			\item In addition, what did you find most challenging professionally? How did you overcome it?	
		\end{itemize}	
    \item \textbf{Individual} and \textbf{group} requirement - Correct spelling and grammar.
    \item \textbf{Individual} requirement - Your \textbf{Git commit messages} should:
    \begin{itemize}
      \item Reflect the context of each functional requirement change.
      \item Be formatted using an appropriate naming convention style.
    \end{itemize}
\end{itemize} 

\subsection*{Additional Information}
\begin{itemize}
    \item \textbf{Do not} rewrite your \textbf{Git} history. It is important that the course lecturer can see how you worked on your assessment over time. 
    \item You need to show the course lecturer the initial \textbf{GitHub} project board before you start your development work. Following this, you need to show the course lecturer your \textbf{GitHub} project board at the end of each week.
\end{itemize} 
\end{document}