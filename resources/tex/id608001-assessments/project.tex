% Author: Grayson Orr
% Course: ID608001: Intermediate Application Development Concepts

\documentclass{article}
\author{}
 
\usepackage{fontspec}
\setmainfont{Arial}

\usepackage{graphicx}
\usepackage{wrapfig}
\usepackage{enumerate}
\usepackage{hyperref}
\usepackage[margin = 1.75cm]{geometry}
\usepackage[table]{xcolor}
\usepackage{fancyhdr}
\hypersetup{
  colorlinks = true,
  urlcolor = blue 
} 
\setlength\parindent{0pt}
\pagestyle{fancy}
\fancyhf{}
\rhead{College of Engineering, Construction and Living Sciences\\Bachelor of Information Technology}
\lfoot{Practical \\Version 1, Semester One, 2024}
\rfoot{\thepage}
 
\begin{document}

\begin{figure}
	\centering
	\includegraphics[width=50mm]{../../../resources/img/logo.png}
\end{figure}

\title{College of Engineering, Construction and Living Sciences\\Bachelor of Information Technology\\ID608001: Intermediate Application Development Concepts\\Level 6, Credits 15\\\textbf{Project}}
\date{}
\maketitle

\section*{Assessment Overview}
In this \textbf{individual} assessment, you will design and develop \textbf{two applications}. The first application needs to be a \textbf{game} using \textbf{Unreal Engine}. The second application can be \textbf{your choice}, i.e., a game, mobile application or web application. In addition, marks will be allocated for code quality and best practices and Git usage.

\section*{Learning Outcomes}
At the successful completion of this course, learners will be able to:
\begin{enumerate}
	\item Apply design patterns and programming principles using software development best practices.
	\item Design and implement full-stack applications using industry relevant programming languages.
\end{enumerate}

\section*{Assessments}
\renewcommand{\arraystretch}{1.5}
\begin{tabular}{|c|c|c|c|}
	\hline
	\textbf{Assessment}                                 & \textbf{Weighting} & \textbf{Due Date}            & \textbf{Learning Outcome} \\ \hline
	\small Practical & \small 20\%        & \small 21-06-2024 (Friday at 4.59 PM)   & \small 1                   \\ \hline
	\small Project                 & \small 80\%        & \small 21-06-2024 (Friday at 4.59 PM) \small  & \small 1 and 2                   \\ \hline
\end{tabular}

\section*{Conditions of Assessment}
You will complete this assessment during your learner-managed time. However, there will be time during class to discuss the requirements and your progress on this assessment. This assessment will need to be completed by \textbf{Friday, 21 June 2024} at \textbf{4.59 PM}. 

\section*{Pass Criteria}
This assessment is criterion-referenced (CRA) with a cumulative pass mark of \textbf{50\%} over all assessments in \textbf{ID608001: Intermediate Application Development Concepts}.

\section*{Authenticity}
All parts of your submitted assessment \textbf{must} be completely your work. Do your best to complete this assessment without using an \textbf{AI generative tool}. You need to demonstrate to the course lecturer that you can meet the learning outcome(s) for this assessment. \\
 
 However, if you get stuck, you can use an \textbf{AI generative tool} to help you get unstuck, permitting you to acknowledge that you have used it. In the assessment's repository \textbf{README.md} file, please include what prompt(s) you provided to the \textbf{AI generative tool} and how you used the response(s) to help you with your work. It also applies to code snippets retrieved from \textbf{StackOverflow} and \textbf{GitHub}. \\
 
 Failure to do this may result in a mark of \textbf{zero} for this assessment.

\section*{Policy on Submissions, Extensions, Resubmissions and Resits}
The school's process concerning submissions, extensions, resubmissions and resits complies with \textbf{Otago Polytechnic | Te Pūkenga} policies. Learners can view policies on the \textbf{Otago Polytechnic | Te Pūkenga} website located at \href{https://www.op.ac.nz/about-us/governance-and-management/policies}{https://www.op.ac.nz/about-us/governance-and-management/policies}.

\section*{Submission}
You \textbf{must} submit all application files via \textbf{GitHub Classroom}. Here is the URL to the repository you will use for your submission – \href{https://classroom.github.com/a/P-HAHGPz}{https://classroom.github.com/a/P-HAHGPz}. If you do not have not one, create a \textbf{.gitignore}. The latest application files in the \textbf{main} branch will be used to mark against the \textbf{Functionality} criterion. Please test before you submit. Partial marks \textbf{will not} be given for incomplete functionality. Late submissions will incur a \textbf{10\% penalty per day}, rolling over at \textbf{5:00 PM}.

\section*{Extensions}
Familiarise yourself with the assessment due date. Extensions will \textbf{only} be granted if you are unable to complete the assessment by the due date because of \textbf{unforeseen circumstances outside your control}. The length of the extension granted will depend on the circumstances and \textbf{must} be negotiated with the course lecturer before the assessment due date. A medical certificate or support letter may be needed. Extensions will not be granted for poor time management or pressure of other assessments.

\section*{Resits}
Resits and reassessments \textbf{are not} applicable in \textbf{ID608001: Intermediate Application Development Concepts}.

\section*{Instructions}

\subsection*{Functionality - Learning Outcomes 1 and 2 (70\%)}
\begin{itemize} 
	\item The topic for the applications is \textbf{your choice}.  
	\item The applications needs to open without code or file structure modification in \textbf{Unreal Engine} and a technology of your choice.
	\item Gather requirements from the client and deconstruct them into user stories.
	\item Design and develop applications using \textbf{Unreal Engine} and a technology of your choice.
	\item Integrate a \textbf{database} into the applications. The type of \textbf{database} is \textbf{your choice}.
	\item Demo the applications on a web platform.
\end{itemize}

\subsection*{Code Quality and Best Practices - Learning Outcome 1 (25\%)}
\begin{itemize}
    \item An appropriate \textbf{.gitignore} file is used. 
    \item Appropriate naming of files, variables, methods and classes.
    \item Idiomatic use of values, control flow, data structures and in-built functions.
    \item Efficient algorithmic approach.
    \item Sufficient modularity.
    \item Each file has an \textbf{comment} located at the top of the file. 
    \item Formatted code.
    \item No dead or unused code.
\end{itemize} 

\subsection*{Git Usage - Learning Outcomes 1 (5\%)}
\begin{itemize}
	\item A \textbf{GitHub} project board or issues to help you organise and prioritise your development work. The course lecturer needs to see consistent use of the \textbf{GitHub} project board or issues for the duration of the assessment.
    \item Your \textbf{Git commit messages} should:
    \begin{itemize}
      \item Reflect the context of each functional requirement change.
      \item Be formatted using an appropriate naming convention style.
    \end{itemize}
\end{itemize}

\subsection*{Additional Information}
\begin{itemize}
    \item \textbf{Do not} rewrite your \textbf{Git} history. It is important that the course lecturer can see how you worked on your assessment over time. 
    \item You need to show the course lecturer the initial \textbf{GitHub} project board or issues before you start your development work. Following this, you need to show the course lecturer your \textbf{GitHub} project board or issues at the end of each week.
\end{itemize} 
\end{document}
\end{document}