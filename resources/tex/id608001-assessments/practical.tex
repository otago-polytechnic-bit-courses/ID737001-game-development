% Author: Grayson Orr
% Course: ID608001: Intermediate Application Development Concepts

\documentclass{article}
\author{}
 
\usepackage{fontspec}
\setmainfont{Arial}

\usepackage{graphicx}
\usepackage{wrapfig}
\usepackage{enumerate}
\usepackage{hyperref}
\usepackage[margin = 1.75cm]{geometry}
\usepackage[table]{xcolor}
\usepackage{fancyhdr}
\hypersetup{
  colorlinks = true,
  urlcolor = blue 
} 
\setlength\parindent{0pt}
\pagestyle{fancy}
\fancyhf{}
\rhead{College of Engineering, Construction and Living Sciences\\Bachelor of Information Technology}
\lfoot{Practical \\Version 1, Semester One, 2024}
\rfoot{\thepage}
 
\begin{document}

\begin{figure}
	\centering
	\includegraphics[width=50mm]{../../../resources/img/logo.png}
\end{figure}

\title{College of Engineering, Construction and Living Sciences\\Bachelor of Information Technology\\ID608001: Intermediate Application Development Concepts\\Level 6, Credits 15\\\textbf{Project}}
\date{}
\maketitle

\section*{Assessment Overview}
In this \textbf{individual} assessment, you will provide documentation that addresses several aspects of the design and development process. In addition, you present 

\section*{Learning Outcomes}
At the successful completion of this course, learners will be able to:
\begin{enumerate}
	\item Apply design patterns and programming principles using software development best practices.
	\item Design and implement full-stack applications using industry relevant programming languages.
\end{enumerate}

\section*{Assessments}
\renewcommand{\arraystretch}{1.5}
\begin{tabular}{|c|c|c|c|}
	\hline
	\textbf{Assessment}                                 & \textbf{Weighting} & \textbf{Due Date}            & \textbf{Learning Outcome} \\ \hline
	\small Practical & \small 20\%        & \small 21-06-2024 (Friday at 4.59 PM)   & \small 1                   \\ \hline
	\small Project                 & \small 80\%        & \small 21-06-2024 (Friday at 4.59 PM) \small  & \small 1 and 2                   \\ \hline
\end{tabular}

\section*{Conditions of Assessment}
You will complete this assessment during your learner-managed time. However, there will be time during class to discuss the requirements and your progress on this assessment. This assessment will need to be completed by \textbf{Friday, 21 June 2024} at \textbf{4.59 PM}. 

\section*{Pass Criteria}
This assessment is criterion-referenced (CRA) with a cumulative pass mark of \textbf{50\%} over all assessments in \textbf{ID608001: Intermediate Application Development Concepts}.

\section*{Authenticity}
All parts of your submitted assessment \textbf{must} be completely your work. Do your best to complete this assessment without using an \textbf{AI generative tool}. You need to demonstrate to the course lecturer that you can meet the learning outcome(s) for this assessment. \\
 
 However, if you get stuck, you can use an \textbf{AI generative tool} to help you get unstuck, permitting you to acknowledge that you have used it. In the assessment's repository \textbf{README.md} file, please include what prompt(s) you provided to the \textbf{AI generative tool} and how you used the response(s) to help you with your work. It also applies to code snippets retrieved from \textbf{StackOverflow} and \textbf{GitHub}. \\
 
 Failure to do this may result in a mark of \textbf{zero} for this assessment.

\section*{Policy on Submissions, Extensions, Resubmissions and Resits}
The school's process concerning submissions, extensions, resubmissions and resits complies with \textbf{Otago Polytechnic | Te Pūkenga} policies. Learners can view policies on the \textbf{Otago Polytechnic | Te Pūkenga} website located at \href{https://www.op.ac.nz/about-us/governance-and-management/policies}{https://www.op.ac.nz/about-us/governance-and-management/policies}.

\section*{Submission}
You \textbf{must} submit all files via \textbf{GitHub Classroom}. Here is the URL to the repository you will use for your submission – \href{https://classroom.github.com/a/P-HAHGPz}{https://classroom.github.com/a/P-HAHGPz}. Late submissions will incur a \textbf{10\% penalty per day}, rolling over at \textbf{5:00 PM}.

\section*{Extensions}
Familiarise yourself with the assessment due date. Extensions will \textbf{only} be granted if you are unable to complete the assessment by the due date because of \textbf{unforeseen circumstances outside your control}. The length of the extension granted will depend on the circumstances and \textbf{must} be negotiated with the course lecturer before the assessment due date. A medical certificate or support letter may be needed. Extensions will not be granted for poor time management or pressure of other assessments.

\section*{Resits}
Resits and reassessments \textbf{are not} applicable in \textbf{ID608001: Intermediate Application Development Concepts}.

\section*{Instructions}

\subsection*{Documentation - Learning Outcome 1 (50\%)}

In a \textbf{Microsoft Word} document called \textbf{documentation}, explain the following:
\begin{itemize}
	\item Design Patterns
	\begin{itemize}
		\item Explain the design patterns used in both applications. 
		\item For each design pattern, provide a code snippet that demonstrates how it is implemented.
		\item Explain the advantages and disadvantages of the chosen design patterns. 
	\end{itemize}
	\item Programming Principles
	\begin{itemize}
		\item Explain the programming principles used in both applications.
		\item For each programming principle, provide a code snippet that demonstrates how it is implemented.
		\item Explain how the programming principles contribute to code maintainability and readability.
	\end{itemize}
\end{itemize}

\subsection*{Presentation - Learning Outcome 1 (50\%)}
\begin{itemize}
	\item Present both applications via a video recording. In addition, you need to answer the following:
	\begin{itemize}
		\item What tools and technologies did you utilise to streamline your design and development workflow?
		\item What challenges did you encounter during the design and development process and how did you overcome them?
		\item What strategies did you employ to maintain code quality and best practices?
		\item How did you handle testing and debugging of both applications?
	\end{itemize}
	\item The presentation must not exceed \textbf{30 minutes} in length.
	\item Upload your presentation to \textbf{OneDrive}. Include a link to your presentation in your repository's \textbf{README.md} file.
\end{itemize}

\end{document}
