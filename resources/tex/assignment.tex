% Author: Grayson Orr
% Course: ID737001: Game Development

\documentclass{article}
\author{}

\usepackage{fontspec}
\setmainfont{Arial}

\usepackage{graphicx}
\usepackage{wrapfig}
\usepackage{enumerate}
\usepackage{hyperref}
\usepackage[margin = 2.25cm]{geometry}
\usepackage[table]{xcolor}
\usepackage{fancyhdr}
\hypersetup{
  colorlinks = true,
  urlcolor = blue
}
\setlength\parindent{0pt}
\pagestyle{fancy}
\fancyhf{}
\rhead{College of Engineering, Construction and Living Sciences\\Bachelor of Information Technology}
\lfoot{Project: Game Development + Demo\\Version 2, Semester One, 2025}
\rfoot{\thepage}
 
\begin{document}

\begin{figure}
	\centering
	\includegraphics[width=50mm]{../../resources/img/logo.jpg}
\end{figure}

\title{College of Engineering, Construction and Living Sciences\\Bachelor of Information Technology\\ID737001: Game Development\\Level 7, Credits 15\\\textbf{Assignment}}
\date{}
\maketitle

\section*{Assessment Overview}
In this assessment, you will:
\begin{itemize}
	\item Form a \textbf{group of Bachelor of IT and Bachelor of Design} learners to design and develop a game using a game engine of your choice. In addition, marks will be allocated for code quality and best practices, documentation and Git usage.
	\item Produce a report covering what you learned from attending a game development meetup. 
\end{itemize}

\section*{Learning Outcome}
At the successful completion of this course, learners will be able to:
\begin{enumerate}
	\item Design and develop a game using industry standard tools, technologies and practices.
\end{enumerate}

\section*{Assessments}
\renewcommand{\arraystretch}{1.5}
\begin{tabular}{|c|c|c|c|}
	\hline
	\textbf{Assessment}                                 & \textbf{Weighting} & \textbf{Due Date}            & \textbf{Learning Outcome} \\ \hline
	\small Assignment                 & \small 30\%        & \small Monday 9th June at 7.59 AM \small  & \small 1                   \\ \hline
	\small Project: Game Development + Demo                 & \small 70\%        & \small Monday 9th June at 7.59 AM \small  & \small 1                   \\ \hline
\end{tabular}

\section*{Conditions of Assessment}
You will complete this assessment during your learner-managed time. However, there will be time during class to discuss the requirements and your progress on this assessment. This assessment will need to be completed by \textbf{Monday, 9 June 2025} at \textbf{7.59 AM}. 

\section*{Pass Criteria}
This assessment is criterion-referenced (CRA) with a cumulative pass mark of \textbf{50\%} over all assessments in \textbf{ID737001: Game Development}.

\section*{Authenticity}
All parts of your submitted assessment \textbf{must} be completely your work. Do your best to complete this assessment without using an \textbf{AI generative tool}. You need to demonstrate to the course lecturer that you can meet the learning outcome for this assessment. \\
 
Learning to use AI tool is an important skill. While AI tools are powerful, you \textbf{must} be aware of the following:

\begin{itemize}
    \item If you provide an AI tool with a prompt that is not refined enough, it may generate a not-so-useful response
    \item Do not trust the AI tool's responses blindly. You \textbf{must} still use your judgement and may need to do additional research to determine if the response is correct
    \item Acknowledge what AI tool you have used. In the assessment's repository \textbf{README.md} file, please include what prompt(s) you provided to the AI tool and how you used the response(s) to help you with your work
\end{itemize}

It also applies to code snippets retrieved from \textbf{StackOverflow} and \textbf{GitHub}. \\
 
Failure to do this may result in a mark of \textbf{zero} for this assessment.

\section*{Policy on Submissions, Extensions, Resubmissions and Resits}
The school's process concerning submissions, extensions, resubmissions and resits complies with \textbf{Otago Polytechnic} policies. Learners can view policies on the \textbf{Otago Polytechnic} website located at \href{https://www.op.ac.nz/about-us/governance-and-management/policies}{https://www.op.ac.nz/about-us/governance-and-management/policies}.

\section*{Submission}
You \textbf{must} submit all application files via \textbf{GitHub}. Create a repository and add the course lecturer as a collaborator. Late submissions will incur a \textbf{10\% penalty per day}, rolling over at \textbf{8:00 AM}.

\section*{Extensions}
Familiarise yourself with the assessment due date. Extensions will \textbf{only} be granted if you are unable to complete the assessment by the due date because of \textbf{unforeseen circumstances outside your control}. The length of the extension granted will depend on the circumstances and \textbf{must} be negotiated with the course lecturer before the assessment due date. A medical certificate or support letter may be needed. Extensions will not be granted for poor time management or pressure of other assessments.

\section*{Resits}
Resits and reassessments \textbf{are not} applicable in \textbf{ID737001: Game Development}.

\section*{Instructions}

\subsection*{Project - Technical and Professional Proficiency (Individual and Group) - Learning Outcome 1 (50\%)}
\begin{itemize}
	 \item \textbf{Group:}
	\begin{itemize}
		\item The game needs to open without code or file structure modification in the chosen game engine.
		\item Gather requirements and deconstruct them into user stories.
		\item Design and develop a game using the chosen game engine that meets the requirements. 
		\item Demo the game on \textbf{itch.io}.
	\end{itemize}
	\item \textbf{Individual:}
	\begin{itemize}
		\item Contribute a meaningful amount of code to the game. This will be judged by the number of \textbf{Git commits} and the number of lines of code contributed.
		\item Perform the following for each feature that is merged into the \textbf{main} branch of the \textbf{GitHub} repository:
		\begin{itemize}
			\item Code review another team member's code.
			\item Play test the feature and provide feedback to the team member.
		\end{itemize}
		This needs to be documented in the \textbf{GitHub} issue that the feature is associated with.
		\item Communicate with team members. This should be through \textbf{Microsoft Teams}. If you wish to use another communication tool, you need to get approval from the course lecturer. Provide screenshots of your communication in the \textbf{GitHub} repository.
	\end{itemize}
\end{itemize}

\subsection*{Project - Code Quality and Best Practices (Individual) - Learning Outcome 1 (20\%)}
\begin{itemize}
    \item Appropriate naming of files, variables, methods and classes.
    \item Idiomatic use of the programming language and game engine.
    \item Efficient algorithmic approach.
    \item Sufficient modularity considering the \textbf{SOLID principles} and \textbf{design patterns}.
    \item Each file has a \textbf{comment} located at the top of the file. The comment should explain the purpose of the file and the author.
    \item Formatted code.
    \item No dead or unused code.
\end{itemize} 

\subsection*{Project - Git Usage (Individual and Group) - Learning Outcome 1 (10\%)}
\begin{itemize}
	\item \textbf{Group} requirement - \textbf{GitHub} project board or issues to help you organise and prioritise your development work. The course lecturer needs to see consistent use of \textbf{GitHub} issues and the project board for the duration of the assessment. 
	\item \textbf{Individual} requirement - Your \textbf{Git commit messages} should:
    \begin{itemize}
      \item Reflect the context of each functional requirement change.
      \item Be formatted using an appropriate naming convention style. 
    \end{itemize}
\end{itemize}

\subsection*{Project - Documentation (Group) - Learning Outcome 1 (10\%)}
\begin{itemize}
	\item Attend and participate in group meetings with \textbf{Bachelor of IT and Bachelor of Design} learners.
	\item For each meeting, record meeting notes. Each member of the group must take turns recording meeting notes.
	\item In a \textbf{Microsoft Word} document, explain the following:
	\begin{itemize}
		\item Date and time of the meeting.
		\item Who attended the meeting.
		\item Main focus of the meeting.
		\item Discussion points including decisions made and ideas shared.
		\item Tasks assigned to each group member including deadlines if applicable.
		\item Follow-up actions or questions that need to be addressed in future meetings.
	\end{itemize}
\end{itemize}

\subsection*{Game Development Meetup Report (Individual) - Learning Outcome 1 (10\%)}
\begin{itemize}
	\item Attend a game development meetup and in a \textbf{Microsoft Word} document, write a game analysis report covering the following:
	\begin{itemize}
		\item Who were the speakers at the meetup and what were their roles?
		\item Did you learn anything about game development that the \textbf{Game Development} pathway has not covered?
		\item What surprised you the most about the meetup?
		\item Did anything challenge your existing assumptions about game development?
		\item How did the meetup compare to your expectations?
		\item Were there any roles or career paths discussed that you had not considered before?
		\item Has this experience influenced your short-term or long-term career goals?
		\item How did it feel talking to other game developers, whether professionals or learners/students?
		\item Did you ask any questions at the meetup? If not, what stopped you? What would you ask if you could go back?
		\item Would you attend another game development meetup? Why or why not?
	\end{itemize} 
	\item Word guideline: \textbf{1000 words}.
	\item Ensure correct spelling and grammar.
	\item Use \textbf{APA 7th edition} for references and in-line citations.
\end{itemize}

\subsection*{Additional Information}
\begin{itemize}
    \item \textbf{Do not} rewrite your \textbf{Git} history. It is important that the course lecturer can see how you worked on your assessment over time. 
    \item You need to show the course lecturer the initial \textbf{GitHub} project board or issues before you start your development work. Following this, you need to show the course lecturer your \textbf{GitHub} project board or issues at the end of each week.
\end{itemize} 
\end{document}